%%%%%%%%%%%%%%%%%%%%%%%%%%%%%%%%%%%%%%%%%
% Cover Latter Template
% LaTeX Template
%
% Original author:
% Oscar Neiva E. Neto (oscar-neiva.github.io)
%
%%%%%%%%%%%%%%%%%%%%%%%%%%%%%%%%%%%%%%%%%

\documentclass[a4paper]{report} %padrao letterpaper, 10pt
\usepackage[utf8]{inputenc}
\usepackage{amsfonts,amssymb,graphicx,enumerate}
\usepackage[centertags]{amsmath}
\usepackage[lmargin=3cm,rmargin=3cm,tmargin=3cm,bmargin=3cm]{geometry}
\usepackage{identfirst}
\usepackage{setspace,cite}
\doublespacing
\renewcommand{\bibname}{Bibliografia}
%\usepackage[portuguese]{babel}

\newenvironment{dem}[1][Demonstra\c c\~ao]{\textbf{#1:}\

}  {\hfill\rule{1ex}{1ex}}

\begin{document}

%%%%%%%%%%%%%%%%%%%%%%%%%%%%%%%%%%%%%%%%%%%%%%%%%%%%%%%%%%%%%%%%%%%%%%%%%%%%%%%%
\section*{\center{Carta de Apresentação}}
\vspace{-1.3cm}
\subsection*{\center{\small{Oscar Neiva}}}

\vspace{1cm}

Sou graduado em Tecnologia da Informação e Comunicação na FAETERJ Petrópolis. Durante a minha graduação, trabalhei como desenvolvedor de software em linguagem C++ no Laboratório de Sistemas Inteligentes e Robótica - SIRLab. Também durante a minha graduação fiz iniciação científica no Laboratório Nacional de Computação Científica - LNCC na área de controle e simulação de sistemas estocásticos. 

No SIRLab trabalhei com uma equipe de mais quatro pessoas na construção de um time de futebol de robôs autônomos para as competições da IEEE VERY Small Size Soccer. Dentre as diversas partes que constituem o desenvolvimento deste projeto, trabalhei na parte de modelagem, pesquisa e desenvolvimento do sistema e dos controladores de robôs moveis do tipo não holonômico. O meu trabalho no SIRLab foi desenvolvido ao longo de 1 ano e 9 meses e durante esse tempo a minha equipe chegou a conquistar a quarta posição na Competição Latino Americana de Robótica no ano de 2014. O trabalho foi feito sobre a orientação dos professores Alberto Angonese e Eduardo Krempser.

Durante minha iniciação científica no LNCC tive contato com alguns conceitos de: sistemas estocástico, cadeias de Markov, sistemas lineares e outras questões da área de sistemas e controle. Assim, ao final do trabalho fiz uma simulação em Matlab do algoritmo PageRank, responsável por fazer o ranqueamento de páginas do buscador Google. De tal forma que na criação do algoritmo do PageRank eu fiz uso dos diversos conceitos estudados no início do trabalho. O meu trabalho no LNCC foi desenvolvido ao longo de 1 ano e 11 meses e foi agraciado com o prêmio "Destaque na Jornada de Iniciação Científica" na Jornada de iniciação científica do LNCC. O trabalho foi feito sobre a orientação do professor Marcos Todorov.

%%%%%%%%%%%%%%%%%%%%%%%%%%%%%%%%%%%%%%%%%%%%%%%%%%%%%%%%%%%%%%%%%%%%%%%%%%%%%%%%%%%%%%%%%%%%%%%%%%%%%%%%%%%%%%

\end{document}  
